\documentclass{beamer}

\usepackage{hyperref}

\usetheme{Marburg}

\title{Maintainable Embedded Linux Solutions}
\author{Thomas Irgang \and Simone Weiß}
\institute{EASTERHEGG 2024 - RABBIT PROTOTYPING}
\date{March 31, 2024}
\titlegraphic{
    \includegraphics[width=2cm]{assets/logo.png}
}

\newcommand\pro{\item[$+$]}
\newcommand\con{\item[$-$]}

\begin{document}

\begin{frame}
    \titlepage
\end{frame}

\section{Why Linux?}

\begin{frame}{Do I need Linux for my project?}
	\begin{tabular}{cccc}
	&\includegraphics[width=1.9cm]{assets/Pixabay_Arduino_integrated-circuit-441289_1280.jpg} &
	\includegraphics[width=1.9cm]{assets/ESP32.png} & 
	\includegraphics[width=1.9cm]{assets/Raspberry_Pi.png} \\
	&\textbf{Arduino} & \textbf{ESP32 FreeRTOS} & \textbf{SBC Linux} \\
	real-time & + &  + & - \\
	energy & + & o & - \\
	UI & - & - & + \\
	IO & - & o & + \\
	CPU & - & o & + \\
	\end{tabular}
\end{frame}

\begin{frame}{What's the life-time of my project?}
	\begin{columns}
    \column{0.33\textwidth}
        \centering
        \textbf{Experiment}
        \begin{itemize}
        		\item few months
        		\item no maintenance
        		\item some reusability
        \end{itemize}
    \column{0.33\textwidth}
        \centering
        \textbf{Media-Center}
        \begin{itemize}
        		\item some years
        		\item typical IT distribution maintenance
        		\item no reusability needed
        \end{itemize}
    \column{0.34\textwidth}
        \centering
        \textbf{Home Automation}
        \begin{itemize}
        		\item more than 15 years
        		\item maintenance and upgrades
        		\item reusability for mid-term upgrade needed
        \end{itemize}
    \end{columns}
\end{frame}

\section{Embedded Linux}


\begin{frame}{The "golden image" approach}
	\begin{columns}
    \column{0.5\textwidth}
        \centering
        \begin{itemize}
        		\item starting from an existing Linux image, e.g. Raspberry Pi OS,
        		\item install additional required tools,
        		\item and configure the tools
        		\item and the Linux image.
        \end{itemize}
    \column{0.5\textwidth}
        \centering
        \begin{itemize}
        		\pro easy approach
        		\pro maintained by base distribution
        		\con solution and base distribution are mixed
        		\con maintenance out of control
        		\con no support for variants
        		\con huge amount of not needed software
        \end{itemize}
    \end{columns}
\end{frame}

\begin{frame}{The source build toolkit approach}
	\begin{columns}
    \column{0.5\textwidth}
        \centering
        \begin{itemize}
        		\item Yocto or Buildroot
        		\item build all packages from source
        		\item using project specific config and patches
        \end{itemize}
    \column{0.5\textwidth}
        \centering
        \begin{itemize}
        		\pro optimal solution possible
        		\pro minimized amount of resources
        		\pro solution separated and reusable
        		\pro support for variants
        		\con learning curve
        		\con high maintenance effort
        \end{itemize}
    \end{columns}
\end{frame}

\begin{frame}{The remix distribution approach}
	\begin{columns}
		\column{0.5\textwidth}
		\centering
		\begin{itemize}
			\item solution-specific remix of existing distribution
			\item build a custom image using the binary packages of the base distribution
		\end{itemize}
		\column{0.5\textwidth}
		\centering
		\begin{itemize}
			\pro low maintenance effort
			\pro lower amount of resources
			\pro solution separated and reusable
			\pro support for variants
			\con learning curve
			\con limited optimization
		\end{itemize}
	\end{columns}
\end{frame}

\section{Remixing a distribution}

\begin{frame}{Tools for building a remix distribution}
	\begin{tabular}{c|ccc}
		& \textbf{Elbe} & \textbf{Debos} & \textbf{Kiwi-ng} \\
		\hline
		target & Embedded Image & Debian remix & Linux remix \\ 
		format & deb & deb & deb, rpm, arch \\
		cross img & yes & yes & yes \\
		cross pkg & yes & no & no \\
		compiler & yes & no & no \\
	\end{tabular}
\end{frame}

\subsection{Kiwi-ng}

\begin{frame}{Kiwi-ng}
	\begin{itemize}
		\item Appliance builder
	\end{itemize}
	\begin{definition} 
		An appliance is a ready to use image of an operating system including a pre-configured application for a specific use case. 
	\end{definition}
	\begin{itemize}
		\item Supports all major package managers
		\item Image description using XML and config scripts
		\item Docs: \url{https://osinside.github.io/kiwi/}
		\item Source: \url{https://github.com/OSInside/kiwi}
	\end{itemize}
\end{frame}

\begin{frame}{Kiwi-ng: Raspberry Pi image}
	\begin{itemize}
		\item Source: \url{https://github.com/OSInside/kiwi-descriptions/tree/main/ubuntu/aarch64/ubuntu-jammy-rpi}
		\item Steps:
		\begin{itemize}
			\item install \url{https://pypi.org/project/kiwi/}
			\item install \url{https://pypi.org/project/kiwi-boxed-plugin/}
			\item get image description
			\item run image build
			\item copy image to SD card
		\end{itemize}
		\item more details: TODO: add URL
	\end{itemize}
\end{frame}

\subsection{Debos}

\subsection{Elbe}

\section{Example project}

\subsection{Raspberry PI image}

% simple RPi CLI image

\subsection{Embedded project}

% TOOD more involved example project

\section{Evaluation}

% Yocto

% Raspbian

% comp with kiwi

% comp with debos

% comp with elbe


\end{document}
